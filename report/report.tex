% vim: set fenc=utf-8 ts=2 sw=2 sts=2 noet foldmethod=indent:
\documentclass[a4paper,12pt]{article}
\usepackage[T2A]{fontenc}
\usepackage[english,russian]{babel}
\usepackage{enumitem}
\usepackage{array}
\usepackage[colorlinks=true]{hyperref}
\usepackage{listings}
\usepackage[dvipsnames]{xcolor}

\usepackage{FiraMono}
\usepackage{fontspec}
\setmainfont{Liberation Serif}

\usepackage{geometry}
\geometry{a4paper, left=3cm, right=1cm, top=2cm, bottom=2cm}

\newfontfamily{\lstfiramono}[Scale=0.8]{Fira Mono}
\lstset{
	basicstyle=\lstfiramono,
	breakatwhitespace=false,
	breaklines=true, % wrapping
	keepspaces=true,
	columns=fixed,
	keywordstyle=\bfseries\color{green!40!black},
	commentstyle=\itshape\color{purple!40!black},
	identifierstyle=\color{blue},
	stringstyle=\color{orange},
	numbers=left,
	numbersep=7pt,
	numberstyle=\small\color{Gray!90},
	showspaces=false,
	showstringspaces=false,
	showtabs=false,
	tabsize=2
}

\newcommand{\InlineCode}[1]{\colorbox{Gray!25}
{\lstinline[identifierstyle=\color{Red!80}]{#1}}}

\begin{document}

\begin{titlepage}
	\begin{center}
		МИНИСТЕРСТВО ОБРАЗОВАНИЯ И НАУКИ РОССИЙСКОЙ ФЕДЕРАЦИИ
		ФЕДЕРАЛЬНОЕ ГОСУДАРСТВЕННОЕ АВТОНОМНОЕ ОБРАЗОВАТЕЛЬНОЕ УЧРЕЖДЕНИЕ
		ВЫСШЕГО ОБРАЗОВАНИЯ
		«САНКТ-ПЕТЕРБУРГСКИЙ ГОСУДАРСТВЕННЫЙ УНИВЕРСИТЕТ
		АЭРОКОСМИЧЕСКОГО ПРИБОРОСТРОЕНИЯ» \\
		\vspace{1cm}
		КАФЕДРА КОМПЬЮТЕРНЫХ ТЕХНОЛОГИЙ И ПРОГРАММНОЙ ИНЖЕНЕРИИ
	\end{center}

	\vspace{2cm}
	\begin{flushleft}
		КУРСОВОЙ ПРОЕКТ \\
		ЗАЩИЩЕН С ОЦЕНКОЙ \\
		РУКОВОДИТЕЛЬ \\
	\end{flushleft}

	\begin{tabular}{p{4cm} p{0.5cm} p{4cm} p{0.5cm} p{4cm}}
		\centering ст.преп. & & & & \hspace{0.9cm} Поляк М.Д. \\
		\cline{1-1} \cline{3-3} \cline{5-5}
		\centering \tiny{должность, уч. степень, звание} & &
		\centering \tiny{подпись, дата} & &
		\centering \tiny{инициалы, фамилия}
	\end{tabular}

	\begin{center}
		\begin{tabular}{p{13cm}}
			\vspace{8cm} \\
			\begin{center}
				ПОЯСНИТЕЛЬНАЯ ЗАПИСКА \\
				К КУРСОВОМУ ПРОЕКТУ \\
				\vspace{1cm}
				ДИСПЕТЧЕР ЗАДАЧ \\
				\vspace{1cm}
				по дисциплине: ОПЕРАЦИОННЫЕ СИСТЕМЫ И СЕТИ
			\end{center}
		\end{tabular}
	\end{center}

	\vspace{4cm}
	\begin{flushleft}
		РАБОТУ ВЫПОЛНИЛ
	\end{flushleft}

	\begin{tabular}{p{3cm} p{1cm} p{0.5cm} p{3.5cm} p{0.5cm} p{3.5cm}}
		СТУДЕНТ ГР. & 4331 & & & & {} \\
		\cline{2-2} \cline{4-4} \cline{6-6}
		& & & \centering \tiny{подпись, дата}
		& & \centering \tiny{инициалы, фамилия}
	\end{tabular}

	\begin{center}
		\vspace{1cm}
		Санкт-Петербург \\
		2016
	\end{center}
\end{titlepage}


\section{Цель работы}
Знакомство с устройством ядра ОС Linux.
Получение опыта разработки драйвера устройства.


\section{Индивидуальное задание}
Диспетчер задач. Реализовать демон для мониторинга использования
блочных устройств. Необходимо в реальном времени отслеживать число 
и размер операций чтения и записи для блочных устройств.


\section{Техническая документация}
\begin{enumerate}
	\item Сборка проекта: \\
		Необходимо перейти в корневой каталог репозитория и вызвать команду
		\InlineCode{gcc daemon.c -o daemon.exe}. Демон будет скомпилирован.
	\item Запуск проекта: \\
		После того, как демон скомпилирован, его необходимо загрузить,
		для этого надо воспользоваться командой \InlineCode{./daemon.exe}.
	\item Использование демона: \\
		Демон каждые 5 секунд обновляет информацию на экране.
		Информация состоит из имени устройства, данных о количестве операций
		чтения и записи, и данных о количестве считанных секторов.
		Также приведено уточнение, что сектор состоит из 512 байт.
	\item Выключение проекта: \\
		Выключение демона осуществялется командой \InlineCode{^C}, 
		либо уничтожением процесса командой  \InlineCode{kill pid}.
\end{enumerate}


\section{Выводы}
В ходе курсового проекта была изучена реализация демонов и способы
получения информации об использовании блочных устройств. 
В последствии демон был реализован на языке C. Полученный
результат удовлетворяет всем поставленным целям данного проекта.


\newpage
\section{Приложение}
Исходный текст мдемона для Linux:
\lstinputlisting[language=C]{../daemon.c}

\end{document}
