\documentclass[a4paper]{article}
\usepackage[14pt]{extsizes}
\usepackage[utf8]{inputenc}
\usepackage[russian]{babel}
\usepackage{setspace,amsmath}
\usepackage{fancyvrb}
\usepackage{graphicx}
\graphicspath{{pictures/}}
\DeclareGraphicsExtensions{.png,.jpg}
%\DefineShortVerb{\|}
%\usepackage[argument]{graphicx}
%\DeclareGraphicsExtensions{.pdf,.png,.jpg}
\usepackage[colorlinks,urlcolor=blue]{hyperref}
\usepackage[left=20mm, top=15mm, right=15mm, bottom=15mm, nohead, footskip=10mm]{geometry} % настройки полей документа



\begin{document}

\begin{titlepage}
	\begin{center}
		МИНИСТЕРСТВО ОБРАЗОВАНИЯ И НАУКИ РОССИЙСКОЙ ФЕДЕРАЦИИ
		ФЕДЕРАЛЬНОЕ ГОСУДАРСТВЕННОЕ АВТОНОМНОЕ ОБРАЗОВАТЕЛЬНОЕ УЧРЕЖДЕНИЕ
		ВЫСШЕГО ОБРАЗОВАНИЯ
		«САНКТ-ПЕТЕРБУРГСКИЙ ГОСУДАРСТВЕННЫЙ УНИВЕРСИТЕТ
		АЭРОКОСМИЧЕСКОГО ПРИБОРОСТРОЕНИЯ» \\
		\vspace{1cm}
		КАФЕДРА КОМПЬЮТЕРНЫХ ТЕХНОЛОГИЙ И ПРОГРАММНОЙ ИНЖЕНЕРИИ
	\end{center}

	\vspace{1cm}
	\begin{flushleft}
		КУРСОВОЙ ПРОЕКТ \\
		ЗАЩИЩЕН С ОЦЕНКОЙ \\
		РУКОВОДИТЕЛЬ \\
	\end{flushleft}

	\begin{tabular}{p{4cm} p{0.5cm} p{4cm} p{0.5cm} p{4cm}}
		\centering ст.преп. & & & & \hspace{0.9cm} Поляк М.Д. \\
		\cline{1-1} \cline{3-3} \cline{5-5}
		\centering \tiny{должность, уч. степень, звание} & &
		\centering \tiny{подпись, дата} & &
		\centering \tiny{инициалы, фамилия}
	\end{tabular}

	\begin{center}
		\begin{tabular}{p{13cm}}
			\vspace{2.5cm} \\
			\begin{center}
				ПОЯСНИТЕЛЬНАЯ ЗАПИСКА \\
				К КУРСОВОМУ ПРОЕКТУ \\
				\vspace{0.5cm}
				ДИСПЕТЧЕР ЗАДАЧ \\
				\vspace{0.5cm}
				по дисциплине: ОПЕРАЦИОННЫЕ СИСТЕМЫ И СЕТИ
			\end{center}
		\end{tabular}
	\end{center}

	\vspace{2.5cm}
	\begin{flushleft}
		РАБОТУ ВЫПОЛНИЛ
	\end{flushleft}

	\begin{tabular}{p{3cm} p{1cm} p{0.5cm} p{3.5cm} p{0.5cm} p{3.5cm}}
		СТУДЕНТ ГР. & 4331 & & & & {} \\
		\cline{2-2} \cline{4-4} \cline{6-6}
		& & & \centering \tiny{подпись, дата}
		& & \centering \tiny{инициалы, фамилия}
	\end{tabular}

	\begin{center}
		\vspace{1cm}
		Санкт-Петербург \\
		2016
	\end{center}
\end{titlepage}

\newpage

\section{Цель работы}

\normalsize Цель работы: Знакомство с устройством ядра ОС Linux.
Получение опыта разработки драйвера устройства.

\section{Задание(5 вариант)}

Диспетчер задач. Реализовать демон для мониторинга использования
блочных устройств. Необходимо в реальном времени отслеживать число 
и размер операций чтения и записи для блочных устройств.

\section{Техническая документация}
\begin{enumerate}
	\item Сборка проекта: \\
		Скачиваем файлы с репозитория на github при помощи команды: \begin{verbatim}
	git clone https://github.com/gurianoff/IOstatistic/IOstatistic.git
\end{verbatim} 
	\item Запуск проекта: \\
	Шаг 1: Необходимо перейти в корневой каталог репозитория и вызвать команду: \begin{verbatim} gcc daemon.c -o daemon.exe
\end{verbatim}  Демон будет скомпилирован.
	 \\
	Шаг 2: После того, как демон скомпилирован, его необходимо загрузить,
		для этого надо воспользоваться командой: \begin{verbatim} ./daemon.exe 
\end{verbatim}
	Шаг 3: Демон каждые 5 секунд обновляет информацию на экране. 
		Информация состоит из имени устройства, данных о количестве 
		операций чтения и записи, и данных о количестве считанных 
		секторов. Также приведено уточнение, что сектор состоит из 512 байт.
	 \\
	Шаг 4: Выключение демона осуществляется командой \begin{verbatim}^C\end{verbatim}, либо уничтожением процесса командой  \begin{verbatim}kill pid\end{verbatim}
\end{enumerate}
\newpage

\section{Выводы}
В ходе курсового проекта была изучена реализация демонов и способы
получения информации об использовании блочных устройств. 
В последствии демон был реализован на языке C. Полученный
результат удовлетворяет всем поставленным целям данного проекта.
\newpage
\section{Приложение}
\begin{verbatim}
daemon.c:

#include <sys/types.h>
#include <sys/stat.h>
#include <stdio.h>
#include <stdlib.h>
#include <fcntl.h>
#include <errno.h>
#include <unistd.h>
#include <syslog.h>
#include <string.h>

#define DISKSTATS_FILE "/proc/diskstats"

void clrscr(void);

int main(void) {

  /* Наши ID процесса и сессии */
  pid_t pid, sid;

  /* Ответвляемся от родительского процесса */
  pid = fork();
  if (pid < 0) {
    exit(EXIT_FAILURE);
  }
  /* Если с PID'ом все получилось, то родительский процесс можно завершить. */
  if (pid > 0) {
      clrscr();
    while(1) {	
	printf("%8s","NameDev");
	printf("%8s", "NumRe");/*kol-vo*/	
	printf("%8s", "SecRe");/*razmer*/
	printf("%8s", "NumWr");/*kol-vo*/	
	printf("%8s \n", "SecWr");/*razmer*/
      	struct device {
		int majorNumber;
		int minorNumber;
		char deviceName[6];
		int readsCompletedSuccessfully;
		int readsMerged;
		int sectorsRead;
		int timeSpentReading;
		int writesCompleted;
		int writesMerged;
		int sectorsWritten;
		int timeSpentWriting;
		int IOsCurrentlyInProgress;
		int timeSpentDoingIOs;
		int weightedTimeSpentDoingIOs;	
	};
	FILE * file = fopen(DISKSTATS_FILE,"r");
	FILE * fileCopy = fopen(DISKSTATS_FILE,"r");
	int lines_count = 0;
	//s4itaem kol-vo strok
	if (file==NULL)
		printf("Trouble of reading file!");
	else {
	while(!feof(file))
	{
		if(fgetc(file)=='\n')
			lines_count++;
	}
	lines_count++;
        fclose(file);	
	//kajdaia stroka iz faila eto structura soderjawaia polia
	struct device block[lines_count];
	char i=0;
	while(fscanf (fileCopy, "%d%d%s%d%d%d%d%d%d%d%d%d%d%d", &(block[i].majorNumber), &(block[i].minorNumber), block[i].deviceName, &(block[i].readsCompletedSuccessfully), &(block[i].readsMerged), &(block[i].sectorsRead), &(block[i].timeSpentReading), &(block[i].writesCompleted),  &(block[i].writesMerged), &(block[i].sectorsWritten), &(block[i].timeSpentWriting), &(block[i].IOsCurrentlyInProgress), &(block[i].timeSpentDoingIOs), &(block[i].weightedTimeSpentDoingIOs)) != EOF) {
		
	//if(block[i].majorNumber==11){//comment this line you need all devices
	printf("%6s %8d %8d %6d %6d \n", block[i].deviceName, block[i].readsCompletedSuccessfully, block[i].sectorsRead, block[i].writesCompleted, block[i].sectorsWritten);
//}//comment this line you need all devices
		i++;
	}
      fclose(fileCopy);
	printf("\n");
	printf("Note: Program shows you the number of hits and the number of used sectors.\n");
	printf("One sector is 512 bytes\n");
      sleep(5);
      clrscr();
     }
    }
    exit(1);
  }

  /* Изменяем файловую маску */
  umask(0);

  /* Здесь можно открывать любые журналы */       
  /* Создание нового SID для дочернего процесса */
  sid = setsid();
  if (sid < 0) {
    /* Журналируем любой сбой */
    exit(EXIT_FAILURE);
  }

  /* Изменяем текущий рабочий каталог */
  if ((chdir("/")) < 0) {
    /* Журналируем любой сбой */
    exit(EXIT_FAILURE);
  }

  /* Закрываем стандартные файловые дескрипторы */
  close(STDIN_FILENO);
  close(STDOUT_FILENO);
  close(STDERR_FILENO);

  exit(EXIT_SUCCESS);
}

void clrscr(void){
	printf("\033[2J"); /*clear the entire screen*/
	printf("\033[0;0f"); /*Move cursor to the top*/
}
\end{verbatim}
\end{document} 
\grid
\grid
